\chapter{Tietorakenteet}

\index{tietorakenne@tietorakenne}

\key{Tietorakenne}
on tapa säilyttää tietoa tietokoneen muistissa.
Sopivan tietorakenteen valinta on tärkeää,
koska kullakin rakenteella on omat
vahvuutensa ja heikkoutensa.
Tietorakenteen valinnassa oleellinen kysymys on,
mitkä operaatiot rakenne toteuttaa tehokkaasti.

Tämä luku esittelee keskeisimmät
C++:n standardikirjaston tietorakenteet.
Valmiita tietorakenteita kannattaa käyttää
aina kun mahdollista, 
koska se säästää paljon aikaa toteutuksessa.
Myöhemmin kirjassa tutustumme erikoisempiin
rakenteisiin, joita ei ole valmiina C++:ssa.

\section{Dynaaminen taulukko}

\index{vektori@vektori}
\index{vector@\texttt{vector}}

\key{Dynaaminen taulukko} on taulukko,
jonka kokoa voi muuttaa
ohjelman suorituksen aikana.
C++:n tavallisin dynaaminen taulukko
on \key{vektori} (\texttt{vector}).
Sitä voi käyttää hyvin samalla tavalla
kuin tavallista taulukkoa.

Seuraava koodi luo tyhjän vektorin
ja lisää siihen kolme lukua:

\begin{lstlisting}
vector<int> v;
v.push_back(3); // [3]
v.push_back(2); // [3,2]
v.push_back(5); // [3,2,5]
\end{lstlisting}

Tämän jälkeen vektorin sisältöä voi käsitellä taulukon tavoin:

\begin{lstlisting}
cout << v[0] << "\n"; // 3
cout << v[1] << "\n"; // 2
cout << v[2] << "\n"; // 5
\end{lstlisting}

Funktio \texttt{size} kertoo, montako alkiota vektorissa on.
Seuraava koodi käy läpi ja tulostaa kaikki vektorin alkiot:

\begin{lstlisting}
for (int i = 0; i < v.size(); i++) {
    cout << v[i] << "\n";
}
\end{lstlisting}

\begin{samepage}
Vektorin voi käydä myös läpi lyhyemmin näin:

\begin{lstlisting}
for (auto x : v) {
    cout << x << "\n";
}
\end{lstlisting}
\end{samepage}

Funktio \texttt{back} hakee vektorin viimeisen alkion,
ja funktio \texttt{pop\_back} poistaa vektorin
viimeisen alkion:

\begin{lstlisting}
vector<int> v;
v.push_back(5);
v.push_back(2);
cout << v.back() << "\n"; // 2
v.pop_back();
cout << v.back() << "\n"; // 5
\end{lstlisting}

Vektorin sisällön voi antaa myös sen luonnissa:

\begin{lstlisting}
vector<int> v = {2,4,2,5,1};
\end{lstlisting}

Kolmas tapa luoda vektori on ilmoittaa
vektorin koko ja alkuarvo:

\begin{lstlisting}
// koko 10, alkuarvo 0
vector<int> v(10);
\end{lstlisting}
\begin{lstlisting}
// koko 10, alkuarvo 5
vector<int> v(10, 5);
\end{lstlisting}

Vektori on toteutettu sisäisesti tavallisena taulukkona.
Jos vektorin koko kasvaa ja taulukko jää liian pieneksi,
varataan uusi suurempi taulukko, johon kopioidaan
vektorin sisältö.
Näin tapahtuu kuitenkin niin harvoin, että vektorin
funktion \texttt{push\_back} aikavaativuus on
keskimäärin $O(1)$.

\index{merkkijono@merkkijono}
\index{string@\texttt{string}}

Myös \key{merkkijono} (\texttt{string}) on dynaaminen taulukko,
jota pystyy käsittelemään lähes samaan
tapaan kuin vektoria.
Merkkijonon käsittelyyn liittyy lisäksi erikoissyntaksia
ja funktioita, joita ei ole muissa tietorakenteissa.
Merkkijonoja voi yhdistää toisiinsa \texttt{+}-merkin avulla.
Funktio $\texttt{substr}(k,x)$ erottaa merkkijonosta
osajonon, joka alkaa kohdasta $k$ ja jonka pituus on $x$.
Funktio $\texttt{find}(\texttt{t})$ etsii kohdan,
jossa osajono \texttt{t} esiintyy merkkijonossa.

Seuraava koodi esittelee merkkijonon käyttämistä:

\begin{lstlisting}
string a = "hatti";
string b = a+a;
cout << b << "\n"; // hattihatti
b[5] = 'v';
cout << b << "\n"; // hattivatti
string c = b.substr(3,4);
cout << c << "\n"; // tiva
\end{lstlisting}

\section{Joukkorakenne}

\index{joukko@joukko}
\index{set@\texttt{set}}
\index{unordered\_set@\texttt{unordered\_set}}

\key{Joukko} on tietorakenne,
joka sisältää kokoelman alkioita.
Joukon perusoperaatiot ovat alkion lisäys,
haku ja poisto.

C++ sisältää kaksi
toteutusta joukolle: \texttt{set} ja \texttt{unordered\_set}.
Rakenne \texttt{set} perustuu tasapainoiseen
binääripuuhun, ja sen operaatioiden aikavaativuus
on $O(\log n)$.
Rakenne \texttt{unordered\_set} pohjautuu hajautustauluun,
ja sen operaatioiden aikavaativuus on keskimäärin $O(1)$.

Usein on makuasia, kumpaa joukon toteutusta käyttää.
Rakenteen \texttt{set} etuna on, että se säilyttää
joukon alkioita järjestyksessä ja tarjoaa
järjestykseen liittyviä funktioita,
joita \texttt{unordered\_set} ei sisällä.
Toisaalta \texttt{unordered\_set} on usein nopeampi rakenne.

Seuraava koodi luo lukuja sisältävän joukon ja
esittelee sen käyttämistä.
Funktio \texttt{insert} lisää joukkoon alkion,
funktio \texttt{count} laskee alkion määrän joukossa
ja funktio \texttt{erase} poistaa alkion joukosta.

\begin{lstlisting}
set<int> s;
s.insert(3);
s.insert(2);
s.insert(5);
cout << s.count(3) << "\n"; // 1
cout << s.count(4) << "\n"; // 0
s.erase(3);
s.insert(4);
cout << s.count(3) << "\n"; // 0
cout << s.count(4) << "\n"; // 1
\end{lstlisting}

Joukkoa voi käsitellä muuten suunnilleen samalla tavalla
kuin vektoria, mutta joukkoa ei voi indeksoida
\texttt{[]}-merkinnällä.
Seuraava koodi luo joukon, tulostaa sen 
alkioiden määrän ja käy sitten läpi kaikki alkiot.
\begin{lstlisting}
set<int> s = {2,5,6,8};
cout << s.size() << "\n"; // 4
for (auto x : s) {
    cout << x << "\n";
}
\end{lstlisting}

Tärkeä joukon ominaisuus on,
että tietty alkio voi esiintyä siinä
enintään kerran.
Niinpä funktio \texttt{count} palauttaa aina
arvon 0 (alkiota ei ole joukossa) tai 1 (alkio on joukossa)
ja funktio \texttt{insert} ei lisää alkiota
uudestaan joukkoon, jos se on siellä valmiina.
Seuraava koodi havainnollistaa asiaa:

\begin{lstlisting}
set<int> s;
s.insert(5);
s.insert(5);
s.insert(5);
cout << s.count(5) << "\n"; // 1
\end{lstlisting}

\index{multiset@\texttt{multiset}}
\index{unordered\_multiset@\texttt{unordered\_multiset}}

C++ sisältää myös rakenteet
\texttt{multiset} ja \texttt{unordered\_multiset},
jotka toimivat muuten samalla tavalla kuin \texttt{set}
ja \texttt{unordered\_set},
mutta sama alkio voi esiintyä
monta kertaa joukossa.
Esimerkiksi seuraavassa koodissa
kaikki luvun 5 kopiot lisätään joukkoon:

\begin{lstlisting}
multiset<int> s;
s.insert(5);
s.insert(5);
s.insert(5);
cout << s.count(5) << "\n"; // 3
\end{lstlisting}
Funktio \texttt{erase} poistaa
kaikki alkion esiintymät
\texttt{multiset}-rakenteessa:
\begin{lstlisting}
s.erase(5);
cout << s.count(5) << "\n"; // 0
\end{lstlisting}
Usein kuitenkin tulisi poistaa
vain yksi esiintymä,
mikä onnistuu näin:
\begin{lstlisting}
s.erase(s.find(5));
cout << s.count(5) << "\n"; // 2
\end{lstlisting}

\section{Hakemisto}

\index{hakemisto@hakemisto}
\index{map@\texttt{map}}
\index{unordered\_map@\texttt{unordered\_map}}

\key{Hakemisto} on taulukon yleistys,
joka sisältää kokoelman avain-arvo-pareja.
Siinä missä taulukon avaimet ovat aina peräkkäiset
kokonaisluvut $0,1,\ldots,n-1$,
missä $n$ on taulukon koko,
hakemiston avaimet voivat
olla mitä tahansa tyyppiä
eikä niiden tarvitse olla peräkkäin.

C++ sisältää kaksi toteutusta hakemistolle
samaan tapaan kuin joukolle.
Rakenne
\texttt{map} perustuu
tasapainoiseen binääripuuhun ja sen
alkioiden käsittely vie aikaa $O(\log n)$,
kun taas rakenne
\texttt{unordered\_map} perustuu
hajautustauluun ja sen alkioiden
käsittely vie keskimäärin aikaa $O(1)$.

Seuraava koodi toteuttaa hakemiston,
jossa avaimet ovat merkkijonoja ja
arvot ovat kokonaislukuja:

\begin{lstlisting}
map<string,int> m;
m["apina"] = 4;
m["banaani"] = 3;
m["cembalo"] = 9;
cout << m["banaani"] << "\n"; // 3
\end{lstlisting}

Jos hakemistosta hakee avainta,
jota ei ole siinä,
avain lisätään hakemistoon
automaattisesti oletusarvolla.
Esimerkiksi seuraavassa koodissa
hakemistoon ilmestyy avain ''aybabtu'',
jonka arvona on 0:

\begin{lstlisting}
map<string,int> m;
cout << m["aybabtu"] << "\n"; // 0
\end{lstlisting}
Funktiolla \texttt{count} voi
tutkia, esiintyykö avain hakemistossa:
\begin{lstlisting}
if (m.count("aybabtu")) {
    cout << "avain on hakemistossa";
}
\end{lstlisting}
Seuraava koodi listaa hakemiston
kaikki avaimet ja arvot:
\begin{lstlisting}
for (auto x : m) {
    cout << x.first << " " << x.second << "\n";
}
\end{lstlisting}

\section{Iteraattorit ja välit}

\index{iteraattori@iteraattori}

Monet C++:n standardikirjaston funktiot
käsittelevät tietorakenteiden iteraattoreita
ja niiden määrittelemiä välejä.
\key{Iteraattori} on muuttuja,
joka osoittaa tiettyyn tietorakenteen alkioon.

Usein tarvittavat iteraattorit ovat \texttt{begin}
ja \texttt{end}, jotka rajaavat välin,
joka sisältää kaikki tietorakenteen alkiot.
Iteraattori \texttt{begin} osoittaa
tietorakenteen ensimmäiseen alkioon,
kun taas iteraattori \texttt{end} osoittaa
tietorakenteen viimeisen alkion jälkeiseen kohtaan.
Tilanne on siis tällainen:

\begin{center}
\begin{tabular}{llllllllll}
\{ & 3, & 4, & 6, & 8, & 12, & 13, & 14, & 17 & \} \\
& $\uparrow$ & & & & & & & & $\uparrow$ \\
& \multicolumn{3}{l}{\texttt{s.begin()}} & & & & & & \texttt{s.end()} \\
\end{tabular}
\end{center}

Huomaa epäsymmetria iteraattoreissa:
\texttt{s.begin()} osoittaa tietorakenteen alkioon,
kun taas \texttt{s.end()} osoittaa tietorakenteen ulkopuolelle.
Iteraattoreiden rajaama joukon väli on siis \emph{puoliavoin}.

\subsubsection{Välien käsittely}

Iteraattoreita tarvitsee
C++:n standardikirjaston funktioissa, jotka käsittelevät
tietorakenteen välejä.
Yleensä halutaan käsitellä tietorakenteiden kaikkia
alkioita, jolloin funktiolle annetaan
iteraattorit \texttt{begin} ja \texttt{end}.

Esimerkiksi seuraava koodi järjestää vektorin funktiolla \texttt{sort},
kääntää sitten alkioiden järjestyksen funktiolla \texttt{reverse}
ja sekoittaa lopuksi alkioiden järjestyksen funktiolla \texttt{random\_shuffle}.

\index{sort@\texttt{sort}}
\index{reverse@\texttt{reverse}}
\index{random\_shuffle@\texttt{random\_shuffle}}

\begin{lstlisting}
sort(v.begin(), v.end());
reverse(v.begin(), v.end());
random_shuffle(v.begin(), v.end());
\end{lstlisting}

Samoja funktioita voi myös käyttää tavallisen taulukon
yhteydessä, jolloin iteraattorin sijasta annetaan
osoitin taulukkoon:

\begin{lstlisting}
sort(t, t+n);
reverse(t, t+n);
random_shuffle(t, t+n);
\end{lstlisting}

\subsubsection{Joukon iteraattorit}

Iteraattoreita tarvitsee usein joukon
alkioiden käsittelyssä.
Seuraava koodi määrittelee iteraattorin
\texttt{it}, joka osoittaa joukon \texttt{s} alkuun:
\begin{lstlisting}
set<int>::iterator it = s.begin();
\end{lstlisting}
Koodin voi kirjoittaa myös lyhyemmin näin:
\begin{lstlisting}
auto it = s.begin();
\end{lstlisting}
Iteraattoria vastaavaan joukon alkioon
pääsee käsiksi \texttt{*}-merkinnällä.
Esimerkiksi seuraava koodi tulostaa
joukon ensimmäisen alkion:

\begin{lstlisting}
auto it = s.begin();
cout << *it << "\n";
\end{lstlisting}

Iteraattoria pystyy liikuttamaan
operaatioilla \texttt{++} (eteenpäin)
ja \texttt{---} (taaksepäin).
Tällöin iteraattori siirtyy seuraavaan
tai edelliseen alkioon joukossa.

Seuraava koodi tulostaa joukon kaikki alkiot:

\begin{lstlisting}
for (auto it = s.begin(); it != s.end(); it++) {
    cout << *it << "\n";
}
\end{lstlisting}
Seuraava koodi taas tulostaa joukon
viimeisen alkion:

\begin{lstlisting}
auto it = s.end();
it--;
cout << *it << "\n";
\end{lstlisting}
% Iteraattoria täytyi liikuttaa askel taaksepäin,
% koska se osoitti aluksi joukon viimeisen
% alkion jälkeiseen kohtaan.

Funktio $\texttt{find}(x)$ palauttaa iteraattorin
joukon alkioon, jonka arvo on $x$.
Poikkeuksena jos alkiota $x$ ei esiinny joukossa,
iteraattoriksi tulee \texttt{end}.

\begin{lstlisting}
auto it = s.find(x);
if (it == s.end()) cout << "x puuttuu joukosta";
\end{lstlisting}

Funktio $\texttt{lower\_bound}(x)$ palauttaa
iteraattorin joukon pienimpään alkioon,
joka on ainakin yhtä suuri kuin $x$.
Vastaavasti $\texttt{upper\_bound}(x)$ palauttaa
iteraattorin pienimpään alkioon,
joka on suurempi kuin $x$.
Jos tällaisia alkioita ei ole joukossa,
funktiot palauttavat arvon \texttt{end}.
Näitä funktioita ei voi käyttää
\texttt{unordered\_set}-rakenteessa,
joka ei pidä yllä alkioiden järjestystä.

\begin{samepage}
Esimerkiksi seuraava koodi etsii joukosta
alkion, joka on lähinnä lukua $x$:

\begin{lstlisting}
auto a = s.lower_bound(x);
if (a == s.begin() && a == s.end()) {
    cout << "joukko on tyhjä\n";
} else if (a == s.begin()) {
    cout << *a << "\n";
} else if (a == s.end()) {
    a--;
    cout << *a << "\n";
} else {
    auto b = a; b--;
    if (x-*b < *a-x) cout << *b << "\n";
    else cout << *a << "\n";
}
\end{lstlisting}

Koodi käy läpi mahdolliset tapaukset
iteraattorin \texttt{a} avulla.
Iteraattori
osoittaa aluksi pienimpään alkioon,
joka on ainakin yhtä suuri kuin $x$.
Jos \texttt{a} on samaan aikaan \texttt{begin}
ja \texttt{end}, joukko on tyhjä.
Muuten jos \texttt{a} on \texttt{begin},
sen osoittama alkio on $x$:ää lähin alkio.
Jos taas \texttt{a} on \texttt{end},
$x$:ää lähin alkio on joukon viimeinen alkio.
Jos mikään edellisistä tapauksista ei päde,
niin $x$:ää lähin alkio
on joko $a$:n osoittama alkio tai sitä edellinen alkio.
\end{samepage}

\section{Muita tietorakenteita}

\subsubsection{Bittijoukko}

\index{bittijoukko@bittijoukko}
\index{bitset@\texttt{bitset}}

\key{Bittijoukko} (\texttt{bitset}) on taulukko,
jonka jokaisen alkion arvo on 0 tai 1.
Esimerkiksi
seuraava koodi luo bittijoukon, jossa on 10 alkiota.
\begin{lstlisting}
bitset<10> s;
s[2] = 1;
s[5] = 1;
s[6] = 1;
s[8] = 1;
cout << s[4] << "\n"; // 0
cout << s[5] << "\n"; // 1
\end{lstlisting}

Bittijoukon etuna on, että se vie tavallista
taulukkoa vähemmän muistia,
koska jokainen alkio vie
vain yhden bitin muistia.
Esimerkiksi $n$ bitin tallentaminen
\texttt{int}-taulukkona vie $32n$
bittiä tilaa, mutta bittijoukkona
vain $n$ bittiä tilaa.
Lisäksi bittijoukon sisältöä
voi käsitellä tehokkaasti bittioperaatioilla,
minkä ansiosta sillä voi tehostaa algoritmeja.

Seuraava koodi näyttää toisen tavan
bittijoukon luomiseen:

\begin{lstlisting}
bitset<10> s(string("0010011010"));
cout << s[4] << "\n"; // 0
cout << s[5] << "\n"; // 1
\end{lstlisting}

Funktio \texttt{count} palauttaa
bittijoukon ykkösbittien määrän:

\begin{lstlisting}
bitset<10> s(string("0010011010"));
cout << s.count() << "\n"; // 4
\end{lstlisting}

Seuraava koodi näyttää esimerkkejä
bittioperaatioiden käyttämisestä:
\begin{lstlisting}
bitset<10> a(string("0010110110"));
bitset<10> b(string("1011011000"));
cout << (a&b) << "\n"; // 0010010000
cout << (a|b) << "\n"; // 1011111110
cout << (a^b) << "\n"; // 1001101110
\end{lstlisting}

\subsubsection{Pakka}

\index{pakka@pakka}
\index{deque@\texttt{deque}}

\key{Pakka} (\texttt{deque}) on dynaaminen taulukko,
jonka kokoa pystyy muuttamaan tehokkaasti
sekä alku- että loppupäässä.
Pakka sisältää vektorin tavoin
funktiot \texttt{push\_back}
ja \texttt{pop\_back}, mutta siinä on lisäksi myös funktiot
\texttt{push\_front} ja \texttt{pop\_front},
jotka käsittelevät taulukon alkua.

Seuraava koodi esittelee pakan käyttämistä:

\begin{lstlisting}
deque<int> d;
d.push_back(5); // [5]
d.push_back(2); // [5,2]
d.push_front(3); // [3,5,2]
d.pop_back(); // [3,5]
d.pop_front(); // [5]
\end{lstlisting}

Pakan sisäinen toteutus on monimutkaisempi kuin
vektorissa, minkä vuoksi se on
vektoria raskaampi rakenne.
Kuitenkin lisäyksen ja poiston
aikavaativuus on keskimäärin $O(1)$ molemmissa päissä.

\subsubsection{Pino}

\index{pino@pino}
\index{stack@\texttt{stack}}

\key{Pino} (\texttt{stack}) on tietorakenne,
joka tarjoaa kaksi $O(1)$-aikaista
operaatiota:
alkion lisäys pinon päälle ja alkion
poisto pinon päältä.
Pinossa ei ole mahdollista käsitellä muita
alkioita kuin pinon päällimmäistä alkiota.

Seuraava koodi esittelee pinon käyttämistä:

\begin{lstlisting}
stack<int> s;
s.push(3);
s.push(2);
s.push(5);
cout << s.top(); // 5
s.pop();
cout << s.top(); // 2
\end{lstlisting}
\subsubsection{Jono}

\index{jono@jono}
\index{queue@\texttt{queue}}

\key{Jono} (\texttt{queue}) on kuin pino,
mutta alkion lisäys tapahtuu jonon loppuun
ja alkion poisto tapahtuu jonon alusta.
Jonossa on mahdollista käsitellä vain
alussa ja lopussa olevaa alkiota.

Seuraava koodi esittelee jonon käyttämistä:

\begin{lstlisting}
queue<int> s;
s.push(3);
s.push(2);
s.push(5);
cout << s.front(); // 3
s.pop();
cout << s.front(); // 2
\end{lstlisting}
% 
% Huomaa, että rakenteiden \texttt{stack} ja \texttt{queue}
% sijasta voi aina käyttää rakenteita
% \texttt{vector} ja \texttt{deque}, joilla voi
% tehdä kaiken saman ja enemmän.
% Kuitenkin \texttt{stack} ja \texttt{queue} ovat
% kevyempiä ja hieman tehokkaampia rakenteita,
% jos niiden operaatiot riittävät algoritmin toteuttamiseen.

\subsubsection{Prioriteettijono}

\index{prioriteettijono@prioriteettijono}
\index{keko@keko}
\index{priority\_queue@\texttt{priority\_queue}}

\key{Prioriteettijono} (\texttt{priority\_queue})
pitää yllä joukkoa alkioista.
Sen operaatiot ovat alkion lisäys ja
jonon tyypistä riippuen joko
pienimmän alkion haku ja poisto tai
suurimman alkion haku ja poisto.
Lisäyksen ja poiston aikavaativuus on $O(\log n)$
ja haun aikavaativuus on $O(1)$.

Vaikka prioriteettijonon operaatiot
pystyy toteuttamaan myös \texttt{set}-ra\-ken\-teel\-la,
prioriteettijonon etuna on,
että sen kekoon perustuva sisäinen
toteutus on yksinkertaisempi
kuin \texttt{set}-rakenteen tasapainoinen binääripuu,
minkä vuoksi rakenne on kevyempi ja
operaatiot ovat tehokkaampia.

\begin{samepage}
C++:n prioriteettijono toimii oletuksena niin,
että alkiot ovat järjestyksessä suurimmasta pienimpään
ja jonosta pystyy hakemaan ja poistamaan suurimman alkion.
Seuraava koodi esittelee prioriteettijonon käyttämistä:

\begin{lstlisting}
priority_queue<int> q;
q.push(3);
q.push(5);
q.push(7);
q.push(2);
cout << q.top() << "\n"; // 7
q.pop();
cout << q.top() << "\n"; // 5
q.pop();
q.push(6);
cout << q.top() << "\n"; // 6
q.pop();
\end{lstlisting}
\end{samepage}

Seuraava määrittely luo käänteisen prioriteettijonon,
jossa jonosta pystyy hakemaan ja poistamaan pienimmän alkion:

\begin{lstlisting}
priority_queue<int,vector<int>,greater<int>> q;
\end{lstlisting}

\section{Vertailu järjestämiseen}

Monen tehtävän voi ratkaista tehokkaasti joko
käyttäen sopivia tietorakenteita
tai taulukon järjestämistä.
Vaikka erilaiset ratkaisutavat olisivat kaikki
periaatteessa tehokkaita, niissä voi olla
käytännössä merkittäviä eroja.

Tarkastellaan ongelmaa, jossa
annettuna on kaksi listaa $A$ ja $B$,
joista kummassakin on $n$ kokonaislukua.
Tehtävänä on selvittää, moniko luku
esiintyy kummassakin listassa.
Esimerkiksi jos listat ovat
\[A = [5,2,8,9,4] \hspace{10px} \textrm{ja} \hspace{10px} B = [3,2,9,5],\]
niin vastaus on 3, koska luvut 2, 5
ja 9 esiintyvät kummassakin listassa.
Suoraviivainen ratkaisu tehtävään on käydä läpi
kaikki lukuparit ajassa $O(n^2)$, mutta seuraavaksi
keskitymme tehokkaampiin ratkaisuihin.

\subsubsection{Ratkaisu 1}

Tallennetaan listan $A$ luvut joukkoon
ja käydään sitten läpi listan $B$ luvut ja
tarkistetaan jokaisesta, esiintyykö se myös listassa $A$.
Joukon ansiosta on tehokasta tarkastaa,
esiintyykö listan $B$ luku listassa $A$.
Kun joukko toteutetaan \texttt{set}-rakenteella,
algoritmin aikavaativuus on $O(n \log n)$.

\subsubsection{Ratkaisu 2}

Joukon ei tarvitse säilyttää lukuja
järjestyksessä, joten
\texttt{set}-ra\-ken\-teen sijasta voi
käyttää myös \texttt{unordered\_set}-ra\-ken\-net\-ta.
Tämä on helppo tapa parantaa algoritmin
tehokkuutta, koska
algoritmin toteutus säilyy samana ja vain tietorakenne vaihtuu.
Uuden algoritmin aikavaativuus on $O(n)$.

\subsubsection{Ratkaisu 3}

Tietorakenteiden sijasta voimme käyttää järjestämistä.
Järjestetään ensin listat $A$ ja $B$,
minkä jälkeen yhteiset luvut voi löytää
käymällä listat rinnakkain läpi.
Järjestämisen aikavaativuus on $O(n \log n)$ ja
läpikäynnin aikavaativuus on $O(n)$,
joten kokonaisaikavaativuus on $O(n \log n)$.

\subsubsection{Tehokkuusvertailu}

Seuraavassa taulukossa on mittaustuloksia
äskeisten algoritmien tehokkuudesta,
kun $n$ vaihtelee ja listojen luvut ovat
satunnaisia lukuja välillä $1 \ldots 10^9$:

\begin{center}
\begin{tabular}{rrrr}
$n$ & ratkaisu 1 & ratkaisu 2 & ratkaisu 3 \\
\hline
$10^6$ & $1{,}5$ s & $0{,}3$ s & $0{,}2$ s \\
$2 \cdot 10^6$ & $3{,}7$ s & $0{,}8$ s & $0{,}3$ s \\
$3 \cdot 10^6$ & $5{,}7$ s & $1{,}3$ s & $0{,}5$ s \\
$4 \cdot 10^6$ & $7{,}7$ s & $1{,}7$ s & $0{,}7$ s \\
$5 \cdot 10^6$ & $10{,}0$ s & $2{,}3$ s & $0{,}9$ s \\
\end{tabular}
\end{center}

Ratkaisut 1 ja 2 ovat muuten samanlaisia,
mutta ratkaisu 1 käyttää \texttt{set}-rakennetta,
kun taas ratkaisu 2 käyttää
\texttt{unordered\_set}-rakennetta.
Tässä tapauksessa tällä valinnalla on
merkittävä vaikutus suoritusaikaan,
koska ratkaisu 2 on 4–5 kertaa
nopeampi kuin ratkaisu 1.

Tehokkain ratkaisu on kuitenkin järjestämistä
käyttävä ratkaisu 3, joka on vielä puolet
nopeampi kuin ratkaisu 2.
Kiinnostavaa on, että sekä ratkaisun 1 että
ratkaisun 3 aikavaativuus on $O(n \log n)$,
mutta siitä huolimatta
ratkaisu 3 vie aikaa vain kymmenesosan.
Tämän voi selittää sillä, että
järjestäminen on kevyt
operaatio ja se täytyy tehdä vain kerran
ratkaisussa 3 algoritmin alussa,
minkä jälkeen algoritmin loppuosa on lineaarinen.
Ratkaisu 1 taas pitää yllä monimutkaista
tasapainoista binääripuuta koko algoritmin ajan.